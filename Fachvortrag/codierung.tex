\section{Codierung}
% Was ist eine Codierung
Unter der Codierung eines Individuums versteht man Informationsketten, mit denen sich Individuen eindeutig voneinander unterscheiden lassen.
% Codierung in der Natur
Die Gene eines Menschen können so als dessen Codierung angesehen werden.
% Codierung bei den Evolutionsstrategien
Bei den Evolutionsstrategien wurde auf eine komplizierte Codierung verzichtet und dafür eine sehr kure Codierungsform eingesetzt. Die relevanten Erbinformationen von Individuen werden durch Vektoren reeller Zahlen dargestellt \cite{}. Diese Vektoren werden Chromosome genannt. Jedes Individuum hat ein solches Chromosom.
Eine Menge an Individuen bildet eine Population. Diese Codierungsweise wird in der Grafik \ref{fig:codierung} dargestellt. Darin ist zu sehen, wie eine Menge an Individuuen eine Population bildet. Die Individuen sind als Männchen verbildlicht. Darüber hinaus ist ein Chromosomausschnnitt abgebildet.
Die Wahl für diese Codierungsvariante lässt sich darin begründen, dass die Evolutionsstrategien zu Beginn für ingenieurstechnische Optimierungen verwendet wurden \cite{}.
In diesem Aufgabenfeld werden optimale Systemparameter gesucht. Diese lassen sich sehr gut als Vektoren reeler Zahlen abbilden.
Der durch die Evolutionsstrategien gewählte Codierungansatz wird als phänotypisch orientiert bezeichnet \cite{}. Das bedeutet, dass im Gegensatz zu einer genotypischen Orientierung, die Eigenschaft des Individuums betrachtet wird und nicht die Chromosome selbst.
Die Eigenschaft des Individuums kann durch Qualitätsfunktionen berechnet werden.

\begin{figure}[!htb]
	\centering
	\includegraphics[width=1.\textwidth]{img/codierung/codierung.png}
	\caption{Grafische Darstellung einer Population, von Individuen und eines Chromosoms.}
\label{fig:codierung}
\end{figure}


