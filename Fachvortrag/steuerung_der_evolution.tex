%-- steuerung der evolution

\section{Steuerung der Evolution}

\subsection{$(1+1)$-ES}

Die $(1+1)$-Evolutionsstrategie gilt als einfachste Form der ES. Mit ihr hat Rechenberg 1964 die optimale Einstellung von Gelenkwinkeln einer Gelenkplatte berechnen lassen.

Es handelt sich um eine zweigliedrige Strategie, bei dem es in jeder Evolutionsiteration genau ein Ur-Individuum bzw. Elternindividuum gibt, welches der Erzeugung von genau einem nachkommenden Individuum (Kindindividuum) dient. Dabei wird der Ausgangsvektor analog zum biologischen Prozess der DNS-Selbstverdopplung dupliziert. Das zweite, bislang wertgleiche, Individuum wird anschließend zufällig, allerdings nicht willkürlich modifiziert. Meist wird ein kleiner reeller Wert auf jeden Vektorparameter addiert.

Die beiden, nun unterschiedlichen, Individuen werden nach dem Prinzip \enquote{survival of the fittest} bewertet.
Anhand des Outputs einer Qualitätsfunktion wird verglichen, welcher der beiden Individuen das zielführendere ist und anschließend zur Fortführung der Evolution selektiert werden soll.
Umgangssprachlich spricht man von einem \enquote{Sterben der Schwachen} und einem \enquote{Überleben der Starken}.
Bei gleicher Qualitätsbewertung wird ein Individuum zufällig selektiert.

\subsection{$(\mu + \lambda)$-ES}



\subsection{$(\mu, \lambda)$-ES}



\subsection{Selektionsdruck}



\subsection{$(\mu / p \# \lambda)$-ES}



\subsection{Populationen}



\subsection{Richtwerte nach Rechenberg und Schwefel}



\subsection{Abbruchbedingungen}



\subsection{Beispiele für Fitness-/Qualitätsfunktionen}


