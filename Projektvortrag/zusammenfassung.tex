%-- einleitung

\section{Zusammenfassung}
Im Folgenden werden die Ergebnisse dieser Projektarbeit zusammengefasst, ein Fazit gebildet anschließend mit einem Ausblick beendet.

\subsection{Ergebnisse und Fazit}
Zu Beginn dieser Ausarbeitung wurde beschrieben, welche Ziele zu erreichen sind. Es sollte ein System entwickelt werden, das die Möglichkeiten von Genetischen Algorithmen an dem Beispiel des TSP demonstriert. Für diesen Zweck wurde ein System umgesetzt, das mithilfe einer eigens entwickelten Bibliothek für Genetische Algorithmen Simulationen durchführen kann. In dieser Ausarbeitung wurden zu diesem Zweck die Entwicklungsschritte des Systems dargestellt und wichtige Programmteile ausführSlich erläutert. Darunter waren neben Codeausschnitten auch die Beschreibung des Testsystems und verwendeter Schnittstellen. 
Mit diesem Ansatz konnte eine wiederverwendbares, dokumentiertes und getestetes System entwickelt werden, das die Funktionen der Genetischen Algorithmen leicht nutzbar macht. 
Nach der Entwicklung des Systems wurde als zweites Ziel definiert, zu Untersuchen welche Stellschrauben bei den Genetischen Algorithmen das Resultat in wieweit verbessert beziehungsweise verschlechtert. Dies wurde im Rahmen von Experimenten erreicht. Dabei konnte festgestellt werden, dass bei den Genetischen Algorithmen vor allem die Populationsgröße sowie die Wahl eines geeigneten Crossover-Verfahrens das Ergebnis beeinflusst. Die Mutation nahm bei den Experimenten eine unwichtigere Rolle ein. Mit einem Testdatensatz von 48 Hauptstädten der USA konnte ein Rundlauf nach Vorgaben des TSP generiert werden, der mit lediglich 34373 Meilen 2,4\% schlechter ist als ein optimaler Rundlauf mit 33551 Meilen. \\
Dieses Ergebnis beweist, dass die Genetischen Algorithmen zur Approximierung von Optimierungsproblemen geeignet sind und als Alternative von beispielsweise Neuronalen Netzwerk eingesetzt werden können.
\subsection{Ausblick}
Die durchgeführten Experimente führten, bei dem gegeben Datensatz, zu sehr guten Ergebnissen. Es wäre aber auch lohnenswert zu Untersuchen, ob ähnlich gute Ergebnisse mit anderen Datensätzen generiert werden können. Ein mögliches Vorgehen wäre es, wenn die Anzahl an Städten größer oder kleiner gewählt wird.
Darüber hinaus wäre es Interessant zu untersuchen, ob es Sonderfälle gibt, bei denen die Genetischen Algorithmen mit dem TSP keine guten Ergebnisse erzielen.
%--